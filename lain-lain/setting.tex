\documentclass[aspectratio=169]{beamer}
\setbeamertemplate{frametitle continuation}{}
\usetheme{Madrid}
\hypersetup{pdfpagemode=FullScreen}
\usepackage[indonesian]{babel}
\usepackage{tikz}
\usetikzlibrary{shapes.geometric, arrows}
\usepackage[none]{hyphenat} %untuk menghilangkanpemenggalan kata
\sloppy
\usepackage{ragged2e}
\usepackage{hyperref}
%\setbeamercovered{transparent}

% Perlengkapan pada table
\usepackage{colortbl} %untuk color table
\usepackage{adjustbox} %agar table pas di halaman (tidak lebih)
\usepackage{longtable} %Untuk membuat table lebih panjang
\usepackage{multirow}

\title[$K$-MEANS DAN ALGORITMA GENETIKA]{PENERAPAN K-MEANS DAN ALGORITMA GENETIKA UNTUK
MENYELESAIKAN MTSP}
\subtitle{(Studi Kasus Pada Perjalanan Menuju Seluruh SMA di Kabupaten Probolinggo)}
\author{Muhammad Faiz Nailun Ni'am}
\institute[UNUJA]{Pendidikan Matematika \\Universitas Nurul Jadid}
\date{\today}
\titlegraphic{\includegraphics[width=2cm]{lain-lain/logo}}