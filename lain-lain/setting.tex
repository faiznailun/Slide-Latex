% Tanpa Animasi
\documentclass[aspectratio=169,handout]{beamer}

% Pdf Untuk di Print
\usepackage{pgfpages}
\pgfpagesuselayout{2 on 1}[a4paper, border shrink=5mm]

% Dengan Animasi
%\documentclass[aspectratio=169]{beamer}

%\hypersetup{pdfpagemode=FullScreen} %agar pdf auto fullscreen

\setbeamertemplate{frametitle continuation}{} %untuk menghilangkan penomoran pada allowframebreaks
\usetheme{Madrid}
\usepackage[indonesian]{babel}

% Perlengkapan flowchart
\usepackage{tikz}
\usetikzlibrary{shapes.geometric, arrows}

% Pemenggalan Kata
\usepackage[none]{hyphenat} %tanpa pemenggalan kata
\sloppy %teks rapi

\usepackage{ragged2e}
\usepackage{hyperref}

%\setbeamercovered{transparent} %agar teks transparan

% Perlengkapan pada table
\usepackage{colortbl} %untuk color table
\usepackage{adjustbox} %agar table pas di halaman (tidak lebih)
\usepackage{longtable} %Untuk membuat table lebih panjang
\usepackage{multirow}

% Identitas
\title[$K$-MEANS DAN ALGORITMA GENETIKA]{PENERAPAN K-MEANS DAN ALGORITMA GENETIKA UNTUK
MENYELESAIKAN MTSP}
\subtitle{(Studi Kasus Pada Perjalanan Menuju Seluruh SMA di Kabupaten Probolinggo)}
\author{Muhammad Faiz Nailun Ni'am}
\institute[UNUJA]{Pendidikan Matematika \\Universitas Nurul Jadid}
\date{\today}
\titlegraphic{\includegraphics[width=2cm]{lain-lain/logo}}