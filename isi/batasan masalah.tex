\section{Batasan Masalah}
\begin{frame}
\frametitle{Batasan Masalah}

\begin{block}<1->{Batasan Masalah}
\begin{enumerate}
\item Menggunakan 1 titik asal dan setiap \textit{salesman} akan berangkat dan kembali pada titik kota yang sama.
\item Titik-titik tujuan adalah koordinat lokasi 75 SMA di Kabupaten Probolinggo baik negeri maupun swasta.
\item Tidak ada prioritas sekolah mana saja yang dilalui terlebih dahulu.
\end{enumerate}
\end{block}
\begin{block}<2->{Asumsi}
\begin{enumerate}
\item Setiap titik tujuan diasumsikan selalu terhubung dan berjalan lurus.
\item Titik asal menggunakan koordinat rata-rata dari semua titik \textit{centroid} untuk mengurangi persilangan
\item Jarak yang digunakan adalah jarak \textit{Euclidean distance} (Jarak garis lurus antara 2 titik)
\end{enumerate}
\end{block}
\end{frame}