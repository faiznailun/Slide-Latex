\section{Alur $K$-means dan Algoritma Genetika}

\begin{frame}
\frametitle{Alur $K$-means dan Algoritma Genetika}
\begin{block}{Algoritma $k$-means}
\begin{figure}[H]
\linespread{1}
\centering
%Definisi
\tikzstyle{bulat} = [rectangle, rounded corners, text width=2cm, minimum height=1cm, minimum width=2cm, text centered, draw=black, fill=red!30]
\tikzstyle{jajargenjang} = [trapezium, trapezium left angle=70, trapezium right angle=110, text width=1.5cm, minimum height=1cm, minimum width=1.5cm, text centered, draw=black, fill=blue!30]
\tikzstyle{kotak} = [rectangle, text width=2.5cm, minimum height=1cm, minimum width=2cm, text centered, draw=black, fill=orange!30]
\tikzstyle{belahketupat} = [diamond, text width=2cm, minimum height=1cm, minimum width=2cm, text centered, draw=black, fill=green!30]
\tikzstyle{garis} = [thick,->,>=stealth]

%Gambar
\begin{tikzpicture}
\onslide<1-> \node (1) [bulat] {Mulai};
\onslide<2-> \node (2) [jajargenjang, below of=1, yshift=-0.8cm]{Dataset, tentukan $n$ klaster};
\onslide<3-> \node (3) [kotak, below of=2, yshift=-1.2cm] {Pilih \textit{centroid} secara acak};
\onslide<4-> \node (4) [kotak, right of=3, xshift=+2.5cm] {Hitung \textit{fitness}};
\onslide<5-> \node (5) [kotak, above of=4, yshift=+2.6cm] {Pengelompokan berdasarkan \textit{fitness} terkecil};
\onslide<6-> \node (6) [kotak, right of=5, xshift=+2.5cm] {Memindahkan \textit{centroid} ke tengah area};
\onslide<7-> \node (7) [belahketupat, below of=6, yshift=-2.6cm]{\textit{Centroid} berbeda?};
\onslide<8-> \node (8) [jajargenjang, right of=7, xshift=+2.5cm] {Hasil klaster};
\onslide<9-> \node (9) [bulat, above of=8, yshift=+0.6cm] {Selesai};

\onslide<1-> \draw [garis] (1) -- (2);
\onslide<2-> \draw [garis] (2) -- (3);
\onslide<3-> \draw [garis] (3) -- (4);
\onslide<4-> \draw [garis] (4) -- (5);
\onslide<5-> \draw [garis] (5) -- (6);
\onslide<6-> \draw [garis] (6) -- (7);
\onslide<7-> \draw [garis] (7) -- node[near start, color=black, yshift=+0.5cm]{Ya}(4);
\onslide<7-> \draw [garis] (7) -- node[near start, color=black, yshift=+0.5cm]{Tidak}(8);
\onslide<8-> \draw [garis] (8) -- (9);

\end{tikzpicture}
\end{figure}
\end{block}

\end{frame}

\begin{frame}
\frametitle{Alur $K$-means dan Algoritma Genetika}

\begin{block}{Algoritma genetika}
\begin{figure}[H]
\centering
\linespread{1}
%Definisi
\tikzstyle{bulat} = [rectangle, rounded corners, text width=2cm, minimum height=1cm, minimum width=2cm, text centered, draw=black, fill=red!30]
\tikzstyle{jajargenjang} = [trapezium, trapezium left angle=70, trapezium right angle=110, text width=1.5cm, minimum height=1cm, minimum width=1.5cm, text centered, draw=black, fill=blue!30]
\tikzstyle{kotak} = [rectangle, text width=2cm, minimum height=1cm, minimum width=2cm, text centered, draw=black, fill=orange!30]
\tikzstyle{belahketupat} = [diamond, text width=2cm, minimum height=1cm, minimum width=2cm, text centered, draw=black, fill=green!30]
\tikzstyle{garis} = [thick,->,>=stealth]
%Gambar
\begin{tikzpicture}
\onslide <1-> \node (S) [bulat] {Mulai};
\onslide <2-> \node (in) [jajargenjang, below of=S, yshift=-0.6cm]{Dataset};
\onslide <3-> \node (pop) [kotak, below of=in, yshift=-0.6cm] {Bangkitkan Populasi Awal};
\onslide <4-> \node (fit) [kotak, right of=pop, xshift=+2.5cm] {Hitung \textit{Fitness}};
\onslide <5-> \node (sel) [kotak, above of=fit, yshift=+0.6cm] {Seleksi};
\onslide <6-> \node (cross) [kotak, above of=sel, yshift=+0.6cm] {\textit{Crossover}};
\onslide <7-> \node (mut) [kotak, right of=cross, xshift=+2.5cm] {Mutasi};
\onslide <8-> \node (opt) [belahketupat, below of=mut, yshift=-2.2cm] {Optimal};
\onslide <9-> \node (out) [jajargenjang, right of=opt, xshift=+2cm]{Kromosom Optimal};
\onslide <10-> \node (E) [bulat, above of=out, yshift=+0.6cm] {Selesai};

\onslide <1-> \draw [garis] (S) -- (in);
\onslide <2-> \draw [garis] (in) -- (pop);
\onslide <3-> \draw [garis] (pop) -- (fit);
\onslide <4-> \draw [garis] (fit) -- (sel);
\onslide <5-> \draw [garis] (sel) -- (cross);
\onslide <6-> \draw [garis] (cross) -- (mut);
\onslide <7-> \draw [garis] (mut) -- (opt);
\onslide <8-> \draw [garis] (opt) -- node[near start, color=black, yshift=+0.3cm]{Ya}(out);
\onslide <8-> \draw [garis] (opt) -- node[near start, color=black, yshift=+0.3cm]{Tidak}(fit);
\onslide <9-> \draw [garis] (out) -- (E);
\end{tikzpicture}
\end{figure}
\end{block}
\end{frame}