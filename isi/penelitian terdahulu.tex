\begin{frame}
\frametitle{Penelitian Terdahulu}
\begin{block}<1->{\textit{Applying K-means and Genetic Algorithm for Solving MTSP}}
Membahas tentang persilangan jalur antara tiap \textit{salesman} yang dapat dihindari dengan menggunakan algoritma genetika dan \textit{k}-means yang dapat meminimalisir terjadinya tabrakan antara \textit{salesman}.
\end{block}

\begin{block}<2->{Optimasi \textit{Multiple Travelling Salesman Problem} (M-TSP) pada Penentuan Rute Optimal Penjemputan Penumpang \textit{Travel} Menggunakan Algoritme Genetika}
Membahas permasalahan \textit{salesman} yang akan berangkat dari kantor \textit{travel} menuju ke alamat penjemputan masing-masing penumpang. Pada permasalahan tersebut menggunakan representasi permutasi, proses reproduksi \textit{crossover}, mutasi, dan seleksi.
\end{block}

\begin{block}<3->{Penyelesaian \textit{Multitraveling Salesman Problem} dengan Algoritma Genetika}
Membahas kinerja algoritma genetika berdasarkan jarak minimum dan waktu pemrosesan yang diperlukan untuk 10 kali pengulangan untuk setiap kombinasi kota penjual.
\end{block}
\end{frame}