\begin{frame}[allowframebreaks]
\frametitle{Kesimpulan}

\begin{block}{}
\begin{enumerate}
\item Jalur terpendek menuju seluruh SMA di Kabupaten Probolinggo dapat menggunakan algoritma genetika dan $k$-means dengan pembagian 7 klaster.
\item Jalur terpendek menuju SMA di Kabupaten Probolinggo dengan 7 klaster dapat menghasilkan jarak terpendek yaitu $4,353294644$ satuan koordinat dengan urutan perjalanan sebagai berikut.
\end{enumerate}
\end{block}

\begin{block}{Urutan perjalanan pada klaster A}
$11\rightarrow30\rightarrow29\rightarrow47\rightarrow21\rightarrow72\rightarrow32\rightarrow56\rightarrow13\rightarrow37\rightarrow55\rightarrow36$
\end{block}

\begin{block}{Urutan perjalanan pada klaster B}
$7\rightarrow70\rightarrow66\rightarrow28\rightarrow51\rightarrow8\rightarrow2\rightarrow34\rightarrow22$
\end{block}

\begin{block}{Urutan perjalanan pada klaster C}
$1\rightarrow19\rightarrow73\rightarrow48\rightarrow69\rightarrow35\rightarrow46\rightarrow68\rightarrow25\rightarrow16\rightarrow5\rightarrow14\rightarrow43\rightarrow71\rightarrow53\rightarrow57$
\end{block}

\begin{block}{Urutan perjalanan pada klaster D}
$67\rightarrow58\rightarrow23\rightarrow12\rightarrow20\rightarrow64\rightarrow39\rightarrow31\rightarrow52\rightarrow15$
\end{block}

\begin{block}{Urutan perjalanan pada klaster E}
$26\rightarrow44\rightarrow50\rightarrow42\rightarrow74$
\end{block}

\begin{block}{Urutan perjalanan pada klaster F}
$24\rightarrow63\rightarrow10\rightarrow59\rightarrow60\rightarrow17\rightarrow33\rightarrow9\rightarrow38\rightarrow27\rightarrow6$
\end{block}

\begin{block}{Urutan perjalanan pada klaster G}
$40\rightarrow49\rightarrow54\rightarrow4\rightarrow41\rightarrow3\rightarrow45\rightarrow61\rightarrow18\rightarrow75\rightarrow65\rightarrow62$
\end{block}

\end{frame}