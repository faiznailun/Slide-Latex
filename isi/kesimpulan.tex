\section{Kesimpulan dan Saran}
\begin{frame}
\frametitle{Kesimpulan dan Saran}

\begin{block}<1->{Kesimpulan}
\begin{enumerate}
\item Jalur terpendek menuju seluruh SMA di Kabupaten Probolinggo dapat menggunakan algoritma genetika dan $k$-means dengan pembagian 7 klaster.
\item Jarak yang dihasilkan dengan pembagian klaster tersebut adalah $4,353294644$ satuan koordinat dengan urutan perjalanan sebagaimana tertera pada naskah skripsi.
\end{enumerate}
\end{block}

\begin{block}<2->{Saran}
\begin{enumerate}
\item Mencoba algoritma lain untuk mengetahui metode yang lebih efektif dan untuk mengurangi persilangan jalur antar \textit{salesman}.
\item Menambahkan variabel waktu tempuh, karena dalam penelitian ini hanya variabel jarak saja.
\item Jarak dapat menggunakan jarak asli bukan dengan \textit{Euclidean distance}
\end{enumerate}
\end{block}
\end{frame}